
\documentclass[9pt]{article}
\usepackage[english]{babel}
\usepackage{graphicx}
\usepackage{amsmath}
\usepackage{geometry}
\usepackage{float}
\geometry{legalpaper,  margin=1in}
\usepackage{amsmath,amssymb,graphicx,cite,amsthm,mathrsfs,bm,rotating,pseudocode}

\begin{document}
\section{Fault detection for Non-additive changes }
There are multiple methods in the literature for detecting changes in the system that must of the are focused on additive changes.
\begin{itemize}
\item 
\end{itemize}
In the case of additive changes, changes are left unchanged by the transformation from observations to innovations. Nonadditive changes are more complex, in the sense that the intuitively obvious idea of monitoring deviations either from zero mean or from whiteness in the sequence of innovations is not convenient for solving a nonadditive change detection problem.

The problem of detecting change in Boolean network, is also nonadditive change as the model of the network is changed when one of the genes stucked at $0/1$.
Our previous work was based on windowing residual error and checking the whiteness of the error.
The drawback of this method is that it needs to have large number of observations. as the window length that we used was 150 points.
For this study, I assume we have 100 points and the change is occurred at point 50.



\end{document}